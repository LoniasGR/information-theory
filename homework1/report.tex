\documentclass[a4paper,12pt]{article}

%%%%%%%%%%%%%%%%%%%%%%%%%%%%%%%%%%%%
%% PACKAGES
%%%%%%%%%%%%%%%%%%%%%%%%%%%%%%%%%%%%
\usepackage[main=greek, english]{babel}

\usepackage{amsmath}

% Change format of sections
\usepackage{titlesec}

%%%%%%%%%%%%%%%%%%%%%%%%%%%%%%%%%%%%
%% SETTINGS
%%%%%%%%%%%%%%%%%%%%%%%%%%%%%%%%%%%%

% Make a macro for changing to english
\newcommand{\english}[1]{\foreignlanguage{english}{{#1}}}

% Configure look of sections
\titleformat{\section}
{\normalfont\Large\bfseries}{Άσκηση \thesection}{0.5em}{}

\titleformat{\subsection}
{\normalfont\Large\bfseries}{\arabic{subsection}}{0.5em}{}

\title{1ή Σειρά Ασκήσεων}
\author{Λεωνίδας Αβδελάς}
\date{\today}



\begin{document}

\maketitle

\section{}

\subsection{}
Η εντροπία υπολογίζεται ως:
\begin{align*}
     H(x) & = - \frac{1}{8} \log{\frac{1}{8}} - \frac{1}{2} \log{\frac{1}{2}} - \frac{1}{8} \log{\frac{1}{8}}  - \frac{1}{24} \log{\frac{1}{24}} \\
          & - \frac{1}{24} \log{\frac{1}{24}} - \frac{1}{24} \log{\frac{1}{24}} - \frac{1}{8} \log{\frac{1}{8}}                                  \\
          & = 2.20\;\text{\english{bits}}
\end{align*}

\subsection{}
Για να υπολογίσουμε την τυχαία μεταβλητή $Y=X^3$, πρέπει να προσθέσουμε όλες τις τιμές του $x$ ώστε
για το $g(x)=y$ να ισχύει $p_Y(y) = \sum_{x | g(x)=y} p_X(x)$.

Έτσι:
\begin{align*}
      & p_Y(-27) = p_X(-3) = \frac{1}{8} \\
      & p_Y(-1) = p_X(-1) = \frac{1}{2}  \\
      & p_Y(0) = p_X(0) = \frac{1}{8}    \\
      & p_Y(1) = p_X(1) = \frac{1}{24}   \\
      & p_Y(8) = p_X(2) = \frac{1}{24}   \\
      & p_Y(27) = p_X(3) = \frac{1}{24}  \\
      & p_Y(64) = p_X(4) = \frac{1}{8}
\end{align*}

Αφού οι πιθανότητες δεν αλλάζουν, έχουμε ότι $H(y) = H(x) = 2.20$ \english{bits}.

\subsection{}

Αντίστοιχα υπολογίζουμε την τυχαία μεταβλητή $Z$, με αντίστοιχο τρόπο με πριν:
\begin{align*}
      & p_Z(81) = p_X(-3) + p_X(3) = \frac{1}{8} + \frac{1}{24} = \frac{4}{24} = \frac{1}{6} \\
      & p_Z(1) = p_X(-1) + p_X(1) = \frac{1}{2} + \frac{1}{24} = \frac{13}{24}               \\
      & p_Z(0) = p_X(0) = \frac{1}{8}                                                        \\
      & p_Z(16) = p_X(2) = \frac{1}{24}                                                      \\
      & p_Z(64) = p_X(4) = \frac{1}{8}
\end{align*}

Άρα η εντροπία είναι:
\begin{align*}
     H(z) & = - \frac{1}{6} \log{\frac{1}{6}} - \frac{13}{24} \log{\frac{13}{24}} - \frac{1}{8} \log{\frac{1}{8}} - \frac{1}{24} \log{\frac{1}{24}} - \frac{1}{8} \log{\frac{1}{8}} \\
          & = 1.85\;\text{\english{bits}}
\end{align*}

\section{}

\subsection{}
Μπορούμε να το υπολογίσουμε εύκολα κάνοντας χρήση των ιδιοτήτων της κοινής εντροπίας.
\begin{align*}
     H(X, Y) = & - 0.2 \log{0.2} - 0  - 0.15 \log{0.15} - 0.2 \log{0.2} - 0.05 \log{0.05} \\
               & - 0.2 \log{0.2} - 0.15 \log{0.15} - 0.01 \log{0.01} - 0.04 \log{0.04}    \\
     =         & 2.68 \;\text{\english{bits}}
\end{align*}

\subsection{}

Θα βρούμε τις περιθωριακές τιμές του $X$ και του $Y$ και μετά θα υπολογίσουμε την εντροπία τους.
\begin{itemize}
     \item Για το $X$ είναι $\{0.35, 0.45, 0.2\}$
     \item Για το $Y$ $\{0.55, 0.06, 0.39\}$.
\end{itemize}
Έτσι οι εντροπίες είναι $H(X) = 1.51$ \english{bits} και $H(Y) = 1.24$ \english{bits}.

\subsection{}

Από τον κανόνα της αλυσίδας έχουμε ότι:
\begin{itemize}
     \item $H(X, Y) = H(X) + H(Y|X)$, άρα $H(Y|X) = H(X,Y) - H(X)$
     \item $H(X, Y) = H(Y) + H(X|Y)$, άρα $H(X|Y) = H(X,Y) - H(Y)$
\end{itemize}

Έτσι $H(X|Y) = 2.68 - 1.24 = 1.44$ και $H(Y|X) = 2.68 - 1.51 = 1.17$.

\subsection{}

Για $Y=1$ έχουμε ότι:
\begin{itemize}
     \item $p(X = 1 | Y = 1) = \frac{p(1,1)}{p(Y=1)} = \frac{0.2}{0.55} = 0.36$
     \item $p(X=2 | Y=1) = \frac{p(2,1)}{p(Y=1)} = \frac{0.2}{0.55} = 0.36$
     \item $p(X=3 | Y=1) = \frac{p(3,1)}{p(Y=1)} = \frac{0.15}{0.55} = 0.28$
\end{itemize}

Άρα: 
\begin{equation*}
     H(X|Y=1) = -0.36 \log{0.36} -0.36 \log{0.36} -0.28 \log{0.28} = 1.54 \; \text{\english{bits}}
\end{equation*}

Για $Y=2$ έχουμε ότι:
\begin{itemize}
     \item $p(X = 1 | Y = 2) = \frac{p(1,2)}{p(Y=2)} = 0$
     \item $p(X=2 | Y=2) = \frac{p(2,2)}{p(Y=2)} = \frac{0.05}{0.06} = 0.83$
     \item $p(X=3 | Y=2) = \frac{p(3,2)}{p(Y=2)} = \frac{0.01}{0.06} = 0.17$
\end{itemize}

Άρα: 
\begin{equation*}
     H(X|Y=2) = 0 -0.83 \log{0.83} -0.17 \log{0.17} = 0.65 \; \text{\english{bits}}
\end{equation*}

Για $Y=3$ έχουμε ότι:
\begin{itemize}
     \item $p(X = 1 | Y = 3) = \frac{p(1,3)}{p(Y=3)} = \frac{0.15}{0.39} = 0.11$
     \item $p(X=2 | Y=3) = \frac{p(2,3)}{p(Y=3)} = \frac{0.2}{0.39} = 0.51$
     \item $p(X=3 | Y=3) = \frac{p(3,3)}{p(Y=3)} = \frac{0.04}{0.39} = 0.10$
\end{itemize}

Άρα: 
\begin{equation*}
     H(X|Y=3) = -0.11 \log{0.11} -0.51 \log{0.51} -0.10 \log{0.10} = 1.78 \; \text{\english{bits}}
\end{equation*}

\subsection{}

Αφού γνωρίζουμε και τους δύο όρους, έχουμε $H(Y) - H(Y|X) = 1.24 - 1.17 = 0.07$.

\subsection{}

Έχουμε βρει τις οριακές πιθανότητες παραπάνω, οπότε τώρα μένει μόνο να 
υπολογίσουμε την σχετική εντροπία.

\begin{align*}
     D\left[p_X(x) \| p_Y(x)\right] &= 0.35 \log{\frac{0.35}{0.55}} + 0.45 \log{\frac{0.45}{0.06}}  + 0.2 \log{\frac{0.2}{0.39}} \\
&= 0.887 \;\text{\english{bits}}
\end{align*}

\begin{align*}
     D\left[p_Y(x) \| p_X(x)\right] &= 0.55 \log{\frac{0.55}{0.35}} + 0.06 \log{\frac{0.06}{0.45}}  + 0.39 \log{\frac{0.39}{0.2}} \\
&= 0.559 \;\text{\english{bits}}
\end{align*}

\subsection{}

Έχουμε:
\begin{align*}
     I(X;Y) = &0.2 \log{\frac{0.2}{0.35 \cdot 0.55}} + 0 + 0.15 \log{\frac{0.15}{0.35 \cdot 0.39}} + 0.2 \log{\frac{0.2}{0.45 \cdot 0.55}}\\
      &+ 0.05 \log{\frac{0.05}{0.45 \cdot 0.06}} + 0.2 \log{\frac{0.2}{0.45 \cdot 0.39}} + 0.15 \log{\frac{0.15}{0.2 \cdot 0.55}}\\
      &+ 0.01 \log{\frac{0.01}{0.2 \cdot 0.06}} + 0.04 \log{\frac{0.04}{0.2 \cdot 0.39}} \\
      = &0.078 \;\text{\english{bits}}
\end{align*}

\section{}

\subsection{}
Έχουμε:
\begin{equation*}
     H(X) = - 0.1 \log{0.1} - 0.2 \log{0.2} - 0.3 \log{0.3} - 0.4 \log{0.4} =  1.85 \text{\english{bits}}
\end{equation*}

\begin{equation*}
     H(Y) = - 0.2 \log{0.2} - 0.3 \log{0.3} - 0.4 \log{0.4} - 0.1 \log{0.1} =  1.85 \text{\english{bits}}
\end{equation*}

\begin{equation*}
     H(Y) = - 0.4 \log{0.4} - 0.1 \log{0.1} - 0.2 \log{0.2} - 0.3 \log{0.3} =  1.85 \text{\english{bits}}
\end{equation*}

\subsection{}

\begin{equation*}
     D(p\|q) = 0.1 \log{\frac{0.1}{0.2}} + 0.2 \log{\frac{0.2}{0.3}} + 0.3 \log{\frac{0.3}{0.4}} + 0.4 \log{\frac{0.4}{0.1}} = 0.458 \text{\english{bits}}
\end{equation*}

\begin{equation*}
     D(q \| p) = 0.2 \log{\frac{0.2}{0.1}} + 0.3 \log{\frac{0.3}{0.2}} + 0.4 \log{\frac{0.4}{0.3}} + 0.1 \log{\frac{0.1}{0.4}} = 0.341 \text{\english{bits}}
\end{equation*}

\begin{equation*}
     D(p\|r) = 0.1 \log{\frac{0.1}{0.4}} + 0.2 \log{\frac{0.2}{0.1}} + 0.3 \log{\frac{0.3}{0.2}} + 0.4 \log{\frac{0.4}{0.3}} = 0.341 \text{\english{bits}}
\end{equation*}

\begin{equation*}
     D(p\|r) = 0.4 \log{\frac{0.4}{0.1}} + 0.1 \log{\frac{0.1}{0.2}} + 0.2 \log{\frac{0.2}{0.3}} + 0.3 \log{\frac{0.3}{0.4}} = 0.258 \text{\english{bits}}
\end{equation*}

\subsection{}

Όπως βλέπουμε, για την εντροπία, η τιμή της τυχαίας μεταβλητής δεν έχει σημασία η τιμή της τυχαίας μεταβλητής, αλλά μόνο οι πιθανότητες των ενδεχομένων. Αντίθετα,
για την απόσταση \english{K-L}, οι τιμές επηρεάζουν πλήρως την τιμή της, αφού όπως βλέπουμε αν αλλάξουμε την σειρά των τυχαίων μεταβλητών τότε το αποτέλεσμα αλλάζει.

\end{document}