\documentclass[a4paper,12pt]{article}

%%%%%%%%%%%%%%%%%%%%%%%%%%%%%%%%%%%%
%% PACKAGES
%%%%%%%%%%%%%%%%%%%%%%%%%%%%%%%%%%%%
\usepackage[main=greek, english]{babel}

\usepackage{amsmath}

% Change format of sections
\usepackage{titlesec}

%%%%%%%%%%%%%%%%%%%%%%%%%%%%%%%%%%%%
%% SETTINGS
%%%%%%%%%%%%%%%%%%%%%%%%%%%%%%%%%%%%

% Make a macro for changing to english
\newcommand{\english}[1]{\foreignlanguage{english}{{#1}}}

% Configure look of sections
\titleformat{\section}
{\normalfont\Large\bfseries}{Άσκηση \thesection}{0.5em}{}

\titleformat{\subsection}
{\normalfont\Large\bfseries}{\arabic{subsection}}{0.5em}{}

\title{1ή Σειρά Ασκήσεων}
\author{Λεωνίδας Αβδελάς}
\date{\today}



\begin{document}

\maketitle

\section{}

\subsection{}
Η εντροπία υπολογίζεται ως:
\begin{align*}
     H(x) & = - \frac{1}{8} \log{\frac{1}{8}} - \frac{1}{2} \log{\frac{1}{2}} - \frac{1}{8} \log{\frac{1}{8}}  - \frac{1}{24} \log{\frac{1}{24}} \\
          & - \frac{1}{24} \log{\frac{1}{24}} - \frac{1}{24} \log{\frac{1}{24}} - \frac{1}{8} \log{\frac{1}{8}}                                  \\
          & = 2.20\;\text{\english{bits}}
\end{align*}

\subsection{}
Για να υπολογίσουμε την τυχαία μεταβλητή $Y=X^3$, πρέπει να προσθέσουμε όλες τις τιμές του $x$ ώστε
για το $g(x)=y$ να ισχύει $p_Y(y) = \sum_{x | g(x)=y} p_X(x)$.

Έτσι:
\begin{align*}
      & p_Y(-27) = p_X(-3) = \frac{1}{8} \\
      & p_Y(-1) = p_X(-1) = \frac{1}{2}  \\
      & p_Y(0) = p_X(0) = \frac{1}{8}    \\
      & p_Y(1) = p_X(1) = \frac{1}{24}   \\
      & p_Y(8) = p_X(2) = \frac{1}{24}   \\
      & p_Y(27) = p_X(3) = \frac{1}{24}  \\
      & p_Y(64) = p_X(4) = \frac{1}{8}
\end{align*}

Αφού οι πιθανότητες δεν αλλάζουν, έχουμε ότι $H(y) = H(x) = 2.20$ \english{bits}.

\subsection{}

Αντίστοιχα υπολογίζουμε την τυχαία μεταβλητή $Z$, με αντίστοιχο τρόπο με πριν:
\begin{align*}
      & p_Z(81) = p_X(-3) + p_X(3) = \frac{1}{8} + \frac{1}{24} = \frac{4}{24} = \frac{1}{6} \\
      & p_Z(1) = p_X(-1) + p_X(1) = \frac{1}{2} + \frac{1}{24} = \frac{13}{24}               \\
      & p_Z(0) = p_X(0) = \frac{1}{8}                                                        \\
      & p_Z(16) = p_X(2) = \frac{1}{24}                                                      \\
      & p_Z(64) = p_X(4) = \frac{1}{8}
\end{align*}

Άρα η εντροπία είναι:
\begin{align*}
     H(z) & = - \frac{1}{6} \log{\frac{1}{6}} - \frac{13}{24} \log{\frac{13}{24}} - \frac{1}{8} \log{\frac{1}{8}} - \frac{1}{24} \log{\frac{1}{24}} - \frac{1}{8} \log{\frac{1}{8}} \\
          & = 1.85\;\text{\english{bits}}
\end{align*}

\section{}

\subsection{}
Μπορούμε να το υπολογίσουμε εύκολα κάνοντας χρήση των ιδιοτήτων της κοινής εντροπίας.
\begin{align*}
     H(X, Y) = & - 0.2 \log{0.2} - 0  - 0.15 \log{0.15} - 0.2 \log{0.2} - 0.05 \log{0.05} \\
               & - 0.2 \log{0.2} - 0.15 \log{0.15} - 0.01 \log{0.01} - 0.04 \log{0.04}    \\
     =         & 2.68 \;\text{\english{bits}}
\end{align*}

\subsection{}

Θα βρούμε τις περιθωριακές τιμές του $X$ και του $Y$ και μετά θα υπολογίσουμε την εντροπία τους.
\begin{itemize}
     \item Για το $X$ είναι $\{0.35, 0.45, 0.2\}$
     \item Για το $Y$ $\{0.55, 0.06, 0.39\}$.
\end{itemize}
Έτσι οι εντροπίες είναι $H(X) = 1.51$ \english{bits} και $H(Y) = 1.24$ \english{bits}.

\subsection{}

Από τον κανόνα της αλυσίδας έχουμε ότι:
\begin{itemize}
     \item $H(X, Y) = H(X) + H(Y|X)$, άρα $H(Y|X) = H(X,Y) - H(X)$
     \item $H(X, Y) = H(Y) + H(X|Y)$, άρα $H(X|Y) = H(X,Y) - H(Y)$
\end{itemize}

Έτσι $H(X|Y) = 2.68 - 1.24 = 1.44$ και $H(Y|X) = 2.68 - 1.51 = 1.17$.

\subsection{}

Έχουμε ότι:
\begin{align*}
     H(X | Y = 1) & = - 0.2\log{0.2} - 0.2\log{0.2} - 0.15\log{0.15} = 1.33 \;\text{\english{bits}}     \\
     H(X | Y = 2) & = 0 - 0.05\log{0.05} - 0.01\log{0.01} = 0.28 \;\text{\english{bits}}                \\
     H(X | Y = 3) & = - 0.15\log{0.15} - 0.2 \log{0.2} - 0.04 \log{0.04} = 1.06 \;\text{\english{bits}}
\end{align*}

\subsection{}

Αφού γνωρίζουμε και τους δύο όρους, έχουμε $H(Y) - H(Y|X) = 1.24 - 1.17 = 0.07$.

\subsection{}

\subsection{}
\end{document}